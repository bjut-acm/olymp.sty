% !TEX root = ../../../statements/main.tex
\gdef\thisproblemauthor{示例作者}
\gdef\thisproblemdeveloper{Ivan Kazmenko}
\gdef\thisproblemorigin{\texttt{XXXIII} St. Petersburg State University Championship}
\begin{problem}{一个示例题面}
{\textsl{standard input}}{\textsl{standard output}}
{2 seconds (\textsl{3 seconds for Java})}{256 mebibytes}{}

这里是题面。

\InputFile

这里是输入信息。

\begin{english} % use english line spacing
    The first line of input contains two integers $n$ and $k$ separated by a space
    ($1 \le n \le 100\,000$, $1 \le k \le 10$).
    The second line contains $n$ integers
    $a_1$, $a_2$, $\ldots$, $a_n$ separated by spaces
    ($1 \le a_i \le 10^9$).
\end{english}

\OutputFile

这里是输出信息。

\begin{english} % use english line spacing
    In the only line, print one number: the answer to the problem.
\end{english}

\Examples

\begin{example}
\exmp{
2 1
1 2
}{%
1
}%
\exmp{
3 3
1 2 3
}{%
0
}%
\end{example}

\begin{example}
\exmpfile{01.t}{01.t.a}%
\end{example}

\begin{examplewide}
\exmpfile{01.t}{01.t.a}%
\end{examplewide}

\begin{examplethree}
\exmpfile{01.t}{01.t.a}{Any text here}%
\end{examplethree}

\Explanations

In the first example, we print $1$.

In the second example, after thinking for a while, we are able to print $0$.

\end{problem}
